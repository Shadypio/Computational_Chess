\chapter{Ruolo del gioco degli scacchi nell'Intelligenza Artificiale} %\label{1cap:spinta_laterale}
% [titolo ridotto se non ci dovesse stare] {titolo completo}
%
\begin{citazione}
    Questo capitolo illustra lo stato dell'arte e i lavori presenti in letteratura sugli aspetti di ricerca trattati nel nostro studio. 
    Prima di dare un'occhiata alla varietà degli approcci adottati per affrontare il problema della complessità degli scacchi, è opportuno fare un passo indietro per capire meglio di \textit{che cosa} si parla.
\end{citazione}
\section{Teoria dei giochi}
La \textbf{teoria dei giochi}\cite{colombo2003introduzione} è una branca della matematica che studia l'interazione strategica tra gli individui nell'ambito di un \textbf{gioco}. In questo senso viene considerato gioco uno scenario in cui i contendenti (giocatori) seguono un determinato tipo di comportamento (strategia) al fine di massimizzare la propria vincita. Nel complesso l'esito finale del gioco coincide con il risultato della sequenza delle strategie adottate dai giocatori e dai loro avversari. Chiarito lo scopo e il significato dei giochi, questi vengono divisi in 5 gruppi principali:
\begin{itemize}
    \item \textbf{giochi cooperativi e non cooperativi}: nei giochi cooperativi i giocatori possono formare alleanze per massimizzare le probabilità di vincita. Ciò non è possibile nei giochi non cooperativi;
    \item \textbf{giochi simmetrici e asimmetrici}: nei giochi simmetrici tutti i partecipanti hanno gli stessi obiettivi, ed è fondamentale elaborare strategie vincenti. Nei giochi asimmetrici, invece, i giocatori hanno obiettivi diversi;
    \item \textbf{giochi ad informazione perfetta e ad informazione imperfetta}: nei giochi ad informazione perfetta tutti i giocatori possono vedere le mosse degli altri giocatori. Le mosse sono invece nascoste nei giochi ad informazione imperfetta;
    \item \textbf{giochi simultanei e sequenziali}: nei giochi simultanei i giocatori possono giocare le proprie mosse simultaneamente, piuttosto che a turni come avviene nei giochi sequenziali;
    \item \textbf{giochi a somma zero e a somma non zero}: nei giochi a somma zero la vincita di un giocatore è equilibrata dalla perdita di un altro. Nei giochi a somma non zero, invece, più giocatori possono trarre vantaggio dalle vincite di un altro.
\end{itemize}
Il gioco degli scacchi è pertanto considerato un gioco non cooperativo e simmetrico, poiché le partite sono solitamente disputate tra due singoli avversari che puntano entrambi a vincere la partita. Ciascun giocatore è tenuto a muovere i propri pezzi a turno davanti agli occhi dell'avversario, e alla fine della partita la vincita di un giocatore comporta la perdita dell'altro: queste caratteristiche rendono gli scacchi un gioco ad informazione perfetta, sequenziale e a somma zero. 
\subsection{Teoria dei giochi e Intelligenza Artificiale}
I giochi ritenuti più interessanti nell'ambito dell'Intelligenza Artificiale sono quelli \textbf{a somma zero} con \textbf{informazione perfetta}, solitamente a turni e a due giocatori. Le azioni degli agenti sono pertanto sequenziali, e le loro funzioni di utilità alla fine della partita sono uguali ma di segno opposto (vittoria +1, sconfitta -1). I primi approcci dell'Intelligenza Artificiale allo studio dei giochi si ebbero già nel 1950 proprio con gli scacchi. Da quel punto in poi le macchine hanno pian piano superato le capacità degli esseri umani anche nella dama e in molti altri giochi\cite{schaeffer2007checkers}. L'interesse verso i giochi è dovuto alla \textbf{facilità} nella loro rappresentazione contrapposta alla \textbf{difficoltà} nel risolverli\cite{russell2005intelligenza}. Un gioco può infatti essere definito come un \textbf{problema di ricerca} avente le seguenti caratteristiche:
\begin{itemize}
    \item lo \textbf{stato iniziale}, che specifica la configurazione di partenza del gioco stesso;
    \item il \textbf{giocatore} a cui tocca fare una mossa in un dato stato;
    \item l'\textbf{insieme delle mosse} lecite in un determinato stato;
    \item il \textbf{modello di transizione}, cioè il risultato di una mossa;
    \item il \textbf{test di terminazione} che controlla se la partita è finita oppure no;
    \item la \textbf{funzione obiettivo} che definisce il valore numerico finale per un gioco che termina in un particolare stato per un particolare giocatore.
\end{itemize}
L'\textbf{obiettivo} principale nello studio di questi problemi è dunque la ricerca di una strategia da adottare considerando le mosse dell'avversario; più in particolare si ricerca una strategia \textbf{ottima} che porti ad un risultato almeno pari a quello di qualsiasi altra strategia, assumendo che si stia giocando contro un giocatore infallibile. Tale ricerca si traduce nella visita dell'\textbf{albero di gioco}, ossia un albero in cui i nodi sono gli stati del gioco e gli archi le mosse. La visita dell'albero di gioco degli scacchi è rimandata al Capitolo \ref{cap: design}. 

\section{Breve panoramica sui motori scacchistici}
\subsection{Turochamp}
Come accennato nel capitolo \ref{cap: introduzione}, la complessità degli scacchi è ciò che li rende una delle sfide più interessanti 
dell'Intelligenza Artificiale. Alan Turing, uno dei padri dell'informatica, fu tra i primi ad interessarsi in maniera concreta al problema, progettando 
\textit{Turochamp}\footnote{Ideato nel 1948, \textit{Turochamp} nacque ben prima di un calcolatore che fosse in grado di 
leggere ed eseguire il programma. Ciò portò lo stesso Turing a valutare la "bontà" dell'algoritmo, analizzando le mosse con carta e penna.}, un algoritmo per giocare a scacchi\cite{godena2021eterna}.
La sua debolezza fu però evidenziata da Garri Kasparov in una conferenza del 2012, che mostrò che l'algoritmo era 
in grado di valutare un numero molto limitato di varianti\cite{kasparov2017reconstructing}. Il contributo di \textit{Turochamp}, in ogni caso, gettò delle importanti basi 
per l'evoluzione delle moderne tecniche di ricerca minimax.

\subsection{Deep Blue}
Fu solo nel 1996 che si cominciarono a temere le enormi potenzialità dei computer in ambito scacchistico: in quell'anno fu disputata 
una partita in condizioni normali di torneo tra Kasparov (allora campione del mondo) e \textit{Deep Blue}, un computer progettato 
da IBM appositamente per giocare a scacchi. La partita si concluse con l'abbandono da parte del campione dopo 40 mosse. Nel 1997, in occasione della rivincita, 
Kasparov abbandonò dopo sole 19 mosse\footnote{Alcune mosse di \textit{Deep Blue} risultarono a Kasparov piuttosto creative e incomprensibili,
al punto da sospettare che la macchina stesse ricevendo un supporto umano nel corso della partita. Effettivamente il codice del programma 
fu modificato tra una sfida e l'altra, permettendo alla macchina di non cadere nelle trappole del campione nelle mosse finali della partita.}. 
Questi eventi aprirono le basi ai moderni motori scacchistici, che nel corso degli anni hanno dimostrato di saper tener testa anche 
ai migliori giocatori di scacchi\cite{newborn2012kasparov}.
La potenza computazionale di \textit{Deep Blue} era dovuta al \textbf{parallelismo massivo}: furono utilizzati ben 480 processori (progettati per il gioco degli 
scacchi), che eseguivano un algoritmo scritto in linguaggio C in grado di calcolare 200 milioni di posizioni al secondo. Le funzioni di 
valutazione erano scritte in forma generale, mentre la lista delle aperture fu fornita dai campioni Illescas, Fedorowicz e De Firmian.

\subsection{Stockfish}
\textit{Stockfish} è un motore scacchistico open-source considerato uno dei più forti in assoluto. Correntemente il motore implementa una rete neurale con \textit{deep learning}. La scacchiera viene rappresentata tramite \textbf{bitboard} (una struttura dati in grado di immagazzinare lo stato di ogni casella della scacchiera all’interno di una parola di 64 bit) e la computazione viene supportata da 512 thread. L'algoritmo di ricerca ad albero è implementato con \textbf{potatura alfa-beta}; in tal modo il numero di nodi da ricercare viene drasticamente ridotto sulla base della funzione di valutazione (la valutazione di una mossa termina non appena viene dimostrato che è peggiore di una mossa già valutata in precedenza). Inoltre, mosse come catture vincenti sono valutate prima di altre mosse, riducendo ulteriormente la profondità dell'albero da visitare e garantendo una maggior efficienza dell'algoritmo di ricerca\cite{enwiki:1105171756}.

\subsection{AlphaZero}
L'evoluzione continua delle tecniche di \textbf{Machine Learning} portò il team di Google DeepMind alla realizzazione di un algoritmo basato su tecniche di apprendimento automatico. \textit{AlphaZero} è infatti guidato da una rete neurale convoluzionale addestrata per rinforzo. In questo modo la qualità di una mossa è valutata da un valore numerico di "ricompensa", incoraggiando il motore ad effettuare scelte simili in futuro (viceversa, un valore numerico di "punizione" allontana il motore a considerare determinate mosse). \textit{AlphaZero} fu addestrato per 9 ore su una singola macchina munita di 4 TPU\footnote{Una \textbf{T}ensor \textbf{P}rocessing \textbf{U}nit è un microprocessore per applicazioni specifiche nel campo delle reti neurali.} giocando contro \textit{Stockfish 8}. Effettivamente la natura dell'algoritmo consentì ad \textit{AlphaZero} di giocare senza libro di apertura e chiusura, arrivando a scoprire e giocare autonomamente diverse aperture che a \textit{Stockfish} vennero invece fornite manualmente\cite{silver2018general}.

\section{Considerazioni sui motori scacchistici analizzati}
Il fascino ludico e allo stesso tempo scientifico del gioco degli scacchi ha sin da subito catturato l'attenzione non solo degli appassionati sportivi ma anche di diverse personalità rilevanti in campo informatico. La nascita dei primi motori scacchistici ha favorito la creazione di diverse piattaforme online, tra le quali spiccano \textbf{Chess.com} e \textbf{Lichess}, che offrono, tra le diverse funzionalità, la possibilità di sostenere partite di scacchi contro motori scacchistici di tutto rispetto. L'incessante studio del gioco degli scacchi continua ancora oggi a proporre nuove idee, nuovi motori scacchistici, contribuendo in modo significativo ai miglioramenti di quelli invece già esistenti\cite{sadler2019tcec15}. I due moduli sviluppati nel presente lavoro di tesi si avvicinano ad un background già esistente in maniera semplice e basilare. Si avverte, da un lato, la volontà di sviluppare una piattaforma semplice, pratica e snellita, differente dalle già esistenti, che offra unicamente la possibilità di giocare partite contro un'Intelligenza Artificiale di cui viene fornita un'analisi delle tecniche adottate per la ricerca delle mosse; dall'altro, si intende sviluppare un motore scacchistico con le tecniche essenziali, la cui semplicità lo distingue in maniera netta dai suoi simili, e analizzarne le performance attraverso partite competitive contro altri motori per comprendere quanto sia fondamentale l'approccio tecnico iniziale per la realizzazione dei più noti motori scacchistici.

\newpage