\chapter{Algoritmi noti e motori scacchistici} %\label{1cap:spinta_laterale}
% [titolo ridotto se non ci dovesse stare] {titolo completo}
%
\begin{citazione}
    Questo capitolo illustra lo stato dell'arte e i lavori presenti in letteratura sugli aspetti di ricerca trattati nel nostro studio. 
    Prima di dare un'occhiata alla varietà degli approcci adottati per affrontate il problema della complessità degli scacchi, 
    è opportuno fare un passo indietro per capire meglio di \textit{che cosa} si parla.
\end{citazione}

\section{Teoria dei giochi}
La \textbf{teoria dei giochi} è una disciplina che studia gli ambienti di interazione strategica tra
agenti intelligenti\footnote{Viene definito agente un sistema in grado di percepire il suo ambiente e di agire su di esso. La peculiarità
di un agente intelligente è da ricercare nei risultati delle sue azioni che, a un osservatore comune, sembrerebbero essere prerogativa
dell'intelligenza di un essere umano.}. Per approcciare un problema di questo tipo, l'agente è necessariamente tenuto a trovare
una strategia che tenga in considerazione le possibili azioni dell'avversario. In questo contesto, una strategia specifica si considera
\textbf{ottima} se porta ad un risultato almeno pari a quello di qualsiasi altra strategia; il problema da risolvere è dunque 
quello di capire \textit{come} poter trovare una strategia ottima.

\subsection{Gli scacchi nella teoria dei giochi}
Gli scacchi vengono classificati come gioco a \textbf{somma zero}\footnote{Il risultato finale di un partecipante
è bilanciato dal risultato di un altro partecipante di valore uguale ma di segno opposto.} e con \textbf{informazione imperfetta}\footnote{Gli stati
del gioco non sono sempre espliciti agli agenti.}.

\newpage