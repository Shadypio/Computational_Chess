\chapter{Algoritmi noti e motori scacchistici} %\label{1cap:spinta_laterale}
% [titolo ridotto se non ci dovesse stare] {titolo completo}
%

Prima di addentrarci nella varietà degli approcci adottati per affrontate il problema della complessità degli scacchi, 
è opportuno fare un passo indietro per capire meglio di \textit{che cosa} stiamo parlando.

\section{Teoria dei giochi}
La \textbf{teoria dei giochi} è una disciplina che studia gli ambienti di interazione strategica tra
agenti\footnote{Viene definito agente un sistema in grado di percepire il suo ambiente e di agire su di esso.} razionali
(intelligenti - nel nostro caso) .
\begin{citazione}
Questo capitolo illustra lo stato dell'arte e i lavori presenti in letteratura sugli aspetti di ricerca trattati nel nostro studio. 

\end{citazione}

\newpage