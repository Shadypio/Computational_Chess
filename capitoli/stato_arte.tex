\chapter{Algoritmi noti e motori scacchistici} %\label{1cap:spinta_laterale}
% [titolo ridotto se non ci dovesse stare] {titolo completo}
%
\begin{citazione}
    Questo capitolo illustra lo stato dell'arte e i lavori presenti in letteratura sugli aspetti di ricerca trattati nel nostro studio. 
    Prima di dare un'occhiata alla varietà degli approcci adottati per affrontate il problema della complessità degli scacchi, 
    è opportuno fare un passo indietro per capire meglio di \textit{che cosa} si parla.
\end{citazione}

\section{Teoria dei giochi}
La \textbf{teoria dei giochi} è una disciplina che studia gli ambienti di interazione strategica tra
agenti intelligenti\footnote{Viene definito agente un sistema in grado di percepire il suo ambiente e di agire su di esso. La peculiarità
di un agente intelligente è da ricercare nei risultati delle sue azioni che, a un osservatore comune, sembrerebbero essere prerogativa
dell'intelligenza di un essere umano.}. Per approcciare un problema di questo tipo, l'agente è necessariamente tenuto a trovare
una strategia che tenga in considerazione le possibili azioni dell'avversario. In questo contesto, una strategia specifica si considera
\textbf{ottima} se porta ad un risultato almeno pari a quello di qualsiasi altra strategia; il problema da risolvere è dunque 
quello di capire \textit{come} poter trovare una strategia ottima. Inseriti in questo contesto,
gli scacchi vengono classificati come gioco a \textbf{somma zero} (il risultato finale di un partecipante
è bilanciato dal risultato di un altro partecipante di valore uguale ma di segno opposto) e con \textbf{informazione imperfetta} (gli stati
del gioco non sono sempre espliciti agli agenti).

\section{Scacchi e Intelligenza Artificiale}
Come accennato nel capitolo precedente, la complessità degli scacchi è ciò che li rende uno degli oggetti di studio più interessanti 
dell'Intelligenza Artificiale. Alan Turing fu uno dei primi a interessarsi in maniera concreta al problema, progettando 
\textit{Turochamp}\footnote{Ideato nel 1948, \textit{Turochamp} nacque ben prima di un calcolatore che fosse in grado di 
leggere ed eseguire il programma. Ciò portò lo stesso Turing a valutare la "bontà" dell'algoritmo, analizzando le mosse con carta e penna.}.
La debolezza del programma fu però evidenziata da Garri Kasparov in una conferenza del 2012, mostrando che l'algoritmo era 
in grao di valutare un numero molto limitato di varianti. Il contributo di \textit{Turochamp}, in ogni caso, gettò delle importanti basi 
per l'evoluzione delle moderne tecniche di ricerca \textit{minimax}\footnote{La ricerca di quiescenza è un algoritmo tipicamente utilizzato per estendere la ricerca a nodi instabili
in alberi da gioco. L'idea è quella di "emulare" delle posizioni a una profondità maggiore di quella che si sta considerando, per assicurarsi
che non vi siano trappole nascoste e per ottenere una stima migliore del valore della posizione.}. 
% Citare con Turochamp
\subsection{Deep Blue}
Fu solo nel 1996 che si cominciarono a temere le enormi potenzialità dei computer in ambito scacchistico: in quell'anno fu disputata 
una partita in condizioni normali di torneo tra Kasparov (allora campione del mondo), e \textit{Deep Blue}, un computer progettato 
da IBM appositamente per giocare a scacchi, con abbandono da parte del campione dopo 40 mosse. Nel 1997, in occasione della rivincita, 
Kasparov abbandonò dopo sole 19 mosse\footnote{Alcune mosse di Deep Blue risultavano a Kasparov piuttosto creative e incomprensibili,
al punto da sospettare che la macchina avesse avuto un supporto umano nel corso della partita. Effettivamente il codice del programma 
fu modificato tra una sfida e l'altra, permettendo alla macchina di non cadere nelle trappole del campione nelle mosse finali del gioco}. 
Questi eventi aprirono le basi ai moderni motori scacchistici, che nel corso degli anni hanno dimostrato di saper tener testa anche 
ai migliori giocatori di scacchi.
% Citare Deepblue vs Kasparov
La potenza computazionale di Deep Blue era dovuta al parallelismo massivo: furono utilizzati ben 480 processori (progettati per il gioco degli 
scacchi), che eseguivano un algoritmo scritto in linguaggio C in grado di calcolare 200 milioni di posizioni al secondo. Le funzioni di 
valutazione erano scritte in forma generale, mentre la lista delle aperture fu fornita dai campioni Illescas, Fedorowicz e De Firmian.
\subsection{Stockfish}
\subsection{AlphaZero}






\newpage