\phantomsection
%\addcontentsline{toc}{chapter}{Introduzione}
\chapter{Introduzione}
\section{Contesto applicativo}
Lo studio proposto è incentrato sulla disciplina degli scacchi. Nello specifico vengono passate a rassegna diverse tecniche di 
Intelligenza Artificiale volte a ricercare (e compiere) una mossa valida sulla scacchiera nel corso di una partita regolare.
Vengono resi noti i due differenti approcci adottati per il conseguimento degli obiettivi sopracitati: 
\begin{itemize}
    \item \textbf{Primo approccio}: degli algoritmi di ricerca valutano lo stato corrente della scacchiera ed effettuano la "migliore"
    mossa legale tra quelle disponibili;
    \item \textbf{Secondo approccio}: viene analizzata una rete neurale addestrata che gioca simulando le mosse di un giocatore umano. 
\end{itemize}
\markboth{Introduzione}{}
% [titolo ridotto se non ci dovesse stare] {titolo completo}

\section{Motivazioni e Obiettivi} %\label{1sec:scopo}
Fra i giochi più popolari al mondo, gli scacchi possono essere giocati ovunque (all'aperto, in circolo, online) e la vastità del numero di 
giocatori è stata tale da favorire lo sviluppo di diverse Federazioni (tra le quali, la più importante, la 
\textbf{Fédération Internationale des Échecs - FIDE}) con conseguenti tornei e competizioni in tutto il mondo. 
Le motivazioni della stesura del presente elaborato sono da ricercare nella natura intrinseca del gioco stesso. 
Claude Shannon, ingegnere e matematico statunitens, nel suo \textit{"Programming a Computer for Playing Chess"} fornì una stima di ($10^{120}$)
partite possibili, dimostrando impraticabile l'idea di affrontare il problema con la forza bruta\footnote{Victor Allis, informatico olandese, 
anni dopo stimò la complessità essere di almeno $10^{123}$, "basata su una media del fattore di ramificazione di 35 e una 
durata media di gioco di 80 coppie di mosse".}. Di fronte a tali numeri, non si può non rimanere disorientati e affascinati allo
stesso tempo, tenendo anche in considerazione che il numero di atomi nell'universo è stimato intorno a $10^{80}$. 
%citare wikipedia
Gli obiettivi finali sono dunque da ricercare nelle motivazioni stesse; la vera protagonista del
presente lavoro di tesi è infatti la \textbf{complessità} del gioco degli scacchi, che viene analizzata, studiata e approfondita nei paragrafi
seguenti, non senza un'attenta critica e analisi accurata sui risultati raggiunti.

\section{Risultati}
%da revisionare
Gli algoritmi di ricerca e di apprendimento sfruttati nei due diversi moduli offrono una panoramica generale sulle moderne tecniche di intelligenza
artificiale che non solo vengono applicate su diverse piattaforme ma sono tutt'oggi in continua evoluzione. 
I risultati ottenuti non vogliono aprire nuovi orizzonti a differenti approcci sullo studio, 
ma fanno più da panoramica generale a tecniche già esistenti.

\section{Struttura della tesi}
La trattazione del lavoro di tesi è strutturata secondo il seguente elenco:
\begin{itemize}
    \item \textbf{Introduzione}: viene fornita una panoramica dello studio effettuato, con particolare attenzione alle motivazioni, 
    agli obiettivi e ai risultati del lavoro svolto.
    \item \textbf{Algoritmi noti e motori scacchistici}: l'attenzione viene spostata sulle tecnologie attualmente in uso e sugli studi 
    che hanno portato il gioco degli scacchi ad essere trattato con le tecniche moderne.
    \item \textbf{Design}: si esaminano nel dettaglio la progettazione e l'implementazione del lavoro.
    \item \textbf{Conclusioni}: vengono presentate riflessioni e considerazioni di carattere generale con eventuali riferimenti
    agli sviluppi futuri.
\end{itemize}
