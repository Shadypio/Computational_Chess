\phantomsection
%\addcontentsline{toc}{chapter}{Introduzione}
\chapter{Introduzione}
Lo studio proposto è incentrato sulla disciplina degli scacchi. Nello specifico vengono passate a rassegna diverse tecniche di 
Intelligenza Artificiale volte a ricercare (e compiere) una mossa valida sulla scacchiera nel corso di una partita regolare.
Vengono resi noti i due differenti approcci adottati per il conseguimento degli obiettivi sopracitati: 
\begin{itemize}
    \item \textbf{Primo approccio}: dei "semplici" algoritmi di ricerca valutano lo stato corrente della scacchiera ed effettuano una 
    possibile mossa;
    \item \textbf{Secondo approccio}: viene addestrato un modello di Intelligenza Artificiale che impari a giocare simulando le 
    mosse di un giocatore umano. 
\end{itemize}
\markboth{Introduzione}{}
% [titolo ridotto se non ci dovesse stare] {titolo completo}

\section{Motivazioni e Obiettivi} %\label{1sec:scopo}
Fra i giochi più popolari al mondo, gli scacchi possono essere giocati ovunque (all'aperto, in circolo, online) e la vastità del numero di 
giocatori è stata tale da favorire lo sviluppo di diverse Federazioni (tra le quali, la più importante, la 
\textbf{Fédération Internationale des Échecs - FIDE}) con conseguenti tornei e competizioni in tutto il mondo. 
Le motivazioni della stesura del presente elaborato sono da ricercare nella natura intrinseca del gioco stesso. Gli scacchi rientrano
tra i giochi di strategia più complessi da padroneggiare, con un numero di mosse esponenzialmente elevato ($10^{123}$) rendendo il numero 
di partite pari a circa $10^{10^{50}}$

\section{Risultati}

\section{Struttura della tesi}
La trattazione del lavoro di tesi è strutturata secondo il seguente elenco:
\begin{itemize}
    \item \textbf{Introduzione}: viene fornita una panoramica dello studio effettuato, con particolare attenzione alle motivazioni, 
    agli obiettivi e ai risultati del lavoro svolto.
    \item \textbf{Stato dell'arte}: l'attenzione viene spostata sulle tecnologie e tecniche adottate in uno specifico contesto.
    \item \textbf{Design}: si esaminano nel dettaglio la progettazione e l'implementazione del lavoro.
    \item \textbf{Conclusioni}: vengono presentate riflessioni e considerazioni di carattere generali con eventuali riferimenti
    agli sviluppi futuri.
\end{itemize}
